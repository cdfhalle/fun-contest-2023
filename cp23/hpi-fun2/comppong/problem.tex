\input{template.tex}

\begin{document}

\makeheader

ProPong

Since you heard about a trend at your University to make highly competitive games out of seemingly simple drinking games you created a new upgraded version of the famous Beerpong game. Unlike in the original game, the rules here are a bit different to eliminate every chance of winning the game just by luck. 

The setup consists of a long table with a team on each of the two shorter sides of it. As in the original game, teams take turns throwing a ping pong ball at some cups placed on the table. Every time a team hits a cup the team owning that cup has to drink its content and take the cup from the table. This also holds if someone manages to hit his/her own cup. The team who first manages to throw a ball into all the cups belonging to their opponents wins the game. 

In contrast to the silly luck-based version of the game, the cups are not placed in a triangle on each side of the table but rather in the shape of a rectangle in the table's center. This rectangle contains  $n \cdot m$ cups of which one half belongs to you and the other half to your opponents. The cups are placed in a random permutation which will be changed after each match so that after an arbitrary number of k games the effects of this randomness will be insignificant. 

Since your friends now expect you to perform in this game which might not be totally achievable with your current ping pong ball throwing skills, you have to develop a Strategy. Because of a mysterious ability you developed, you can perfectly guess the number of sips in each cup. To stay as sober as possible, you try to aim for the area of the cup-matrix that has the greatest potential of making your opponents drunk while letting you stay almost sober. To simplify things a bit the size of the area you aim for does not really matter.


\paragraph*{Input}

A single line containing $1\leq k \leq 10^{tbd}$  which denotes the number of games that will be played.
$k$ lines containing the dimensions of the cup matrix ($w h$) in each game $1 \leq w,h \leq 10^{tbd}$, followed by each $h$ rows containing each $w$ numbers $-10^{tbd} \leq x \leq 10^{tbd}$ of which a positive value means that the cup belongs to you and a negative value that it belongs to your opponents.

\paragraph*{Output}

For each game print out four numbers $0\leq x_1 < w, 0\leq y_1 < h, 0\leq x_2 < w, 0 \leq y_2 < h$ denoting the upper left and the lower right corner of the submatrix that contains the biggest sum.
If there are multiple options with the same sum choose the larger area.

\placeholder{Output description}

\begin{samples}
  \sample{sample2}
\end{samples}

% Solution sketch:

% Kadane's algorithm for 1D array: returns the maximum sum and stores starting and ending indexes of the maximum sum subarray at addresses pointed by start and finish pointers respectively.
% int kadane(array, start, finish, n){
%     initialize sum = 0, maxSum = max int 


%     for every field i in row:
%         sum += arr[i];
%         if sum < 0:
%             sum = 0
%             local_start = i + 1
%         else if sum > maxSum:
%             maxSum = sum
%             start = local_start
%             finish = i
%     }

%   If there is at-least one non-negative number: return 
    
%   If all numbers in array are negative: return max element in array
    
    

% finds maximum sum rectangle
% findMaxSum(int M[][COL]):
%     Set the left column
%     for every column (indicated by left which goes from 0 to number of columns COL):
%         Initialize all elements of temp as 0

%         Set the right column for the left column set by outer loop
%         for(right = left; right < COL; ++right) {

%             Calculate sum between current left and right for every row 'i' in temp[i]

%             Find the maximum sum subarray in temp[] with kadane() function. 
%             'sum' is sum of rectangle between (start, left) and (finish, right) which is the maximum sum with boundary 
%             columns strictly as left and right.
           
%             sum = kadane(temp, \&start, \&finish, ROW);

%             Compare this sum withour maximum sum. If new sum > max sum: update max sum

\end{document}
